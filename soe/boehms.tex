\begin{figure}[h]
\centering
\begin{tikzpicture}[scale=1]
\path (0:0cm) coordinate (O); % define coordinate for origin

  % draw the spiderweb
  \foreach \X in {1,...,\D}{
    \draw (\X*\A:0) -- (\X*\A:\R);
  }

  \foreach \Y in {0,...,\U}{
    \foreach \X in {1,...,\D}{
      \path (\X*\A:\Y*\R/\U) coordinate (D\X-\Y);
      \fill (D\X-\Y) circle (1pt);
    }
    \draw [opacity=0.3] (0:\Y*\R/\U) \foreach \X in {1,...,\D}{
        -- (\X*\A:\Y*\R/\U)
    } -- cycle;
  }

  % define labels for each dimension axis (names config option)
  \path (1*\A:\L) node (L1) {\tiny Personnel};
  \path (2*\A:\L) node (L2) {\tiny Criticality};
  \path (3*\A:\L) node (L3) {\tiny Size};
  \path (4*\A:\L) node (L4) {\tiny Culture};
  \path (5*\A:\L) node (L5) {\tiny Dynamism};

  % for each sample case draw a path around the web along concrete values
  % for the individual dimensions. Each node along the path is labeled
  % with an identifier using the following scheme:
  %
  %   D<d>-<v>, dimension <d> a number between 1 and \D (#dimensions) and
  %             value <v> a number between 0 and \U (#scale units)
  %
  % The paths will be drawn half-opaque, so that overlapping parts will be
  % rendered in a composite color.

  % Example Case 3 (blue)
  %
  % D1 (Security): 1/7; D2 (Content Quality): 7/7; D3 (Performance): 4/7;
  % D4 (Stability): 4/7; D5 (Usability): 3/7; D6 (Generality): 5/7;
  % D7 (Popularity): 2/7
  \draw [color=blue,line width=1.5pt,opacity=0.5]
    (D1-1) --
    (D2-1) --
    (D3-1) --
    (D4-2) --
    (D5-4) -- cycle;

\end{tikzpicture}
\caption{Boehms stjerne-model}
\label{fig:boehms}
\end{figure}