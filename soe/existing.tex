% !TeX spellcheck = da_DK
\subsection{Eksisterende metoder}\label{existing}
Til at finde en passende udviklingsmetode til projektgruppen er der kigget p� f�lgende eksisterende agile metoder: SCRUM, XP og UP.\cite{larman}
Der er kun kigget netop disse metoder, da de er blevet pr�senteret i \emph{Software Engineering} kurset som v�rende meget udbredte metoder.
Vi kan derfor med fordel v�lge en af disse metoder til vort studieprojekt.


\paragraph{XP} er en agil metode der tager udgangspunkt i at de udviklere der er i gruppen har et h�jt fagligt niveau.
Med det udgangspunkt frav�lges en r�kke elementer fra plandrevet udvikling der ikke giver kunden direkte v�rdi.
P� et semesterprojekt har vi ikke en egentlig kunde at udvikle til, hvilket p�virker afvejningen af hvad der skaber v�rdi.
Samtidig er vi i en l�ringsproces hvor den bagved liggende teori er af stor relevans, og skal dokumenteres i form af en rapport.
Derfor har vi fravalgt XP som overordnet metode, men vil i \cref{ourmethod} ``\nameref{ourmethod}'' beskrive nogle af de elementer vi inkluderer til programmering.

\paragraph{UP} er en ligeledes en agil metode til software udvikling.
UP er dog baseret p� st�rre teams med udskiftning, og sikrer via forskellige typer dokumentation at denne udskiftning er mulig.
Eftersom det ``team'' vi arbejder i er p� seks mand, og derfor ikke er s�rlig stort (i form af projektgruppen, jf. \cref{boehms:stjerne}) og helt uden udskiftning, har vi helt fravalgt UP.
Vi ser desuden ogs� p� Boehms model at der er en h�j grad af kaos, hvilket strider mod den rigide dokumentation i UP.
\todo[inline]{rigide dokumentation kunne reducres ved kun at benytte nogle udvalgte praksis. Skal vi skrive om det ?}

\paragraph{SCRUM} er en agil udviklingsmetode der best�r af sm� selvstyrede hold med daglige evalueringer af fremskridt. 
I scrum er produktet i fokus, og kunden inddrages ved hver iteration for at sikre at projektet er p� vej i den rigtige retning. Scrum er meget fleksibel og nem at tilpasse, hvilket g�r den let at implementere i et semesterprojekt. 
Vi har derfor valgt at bruge scrum som udgangspunkt for vores udviklingsmetode i dette projekt.
