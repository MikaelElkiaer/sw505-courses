\subsection{Kig p� eksisterende metoder}
Til at finde en passende udviklingsmetode til projektgruppen er der kigget p� f�lgende eksisterende agile metoder: SCRUM, XP og UP.\cite{larman}
Der er kun kigget p� disse metoder, da netod de er blevet pr�senteret i \emph{Software Engineering} kurset som v�rende meget udbredte metoder.
Vi kan derfor med fordel v�lge en af disse metoder til vort studieprojekt.

\paragraph{SCRUM} stemmer godt overens med projektet, da den er meget fleksibel og nem at tilpasse.

\paragraph{XP} passer i mindre grad.
Da vi har faste deadlines (projektaflevering), er id�en med flere releases ikke s� relevant.
Standup-meetings, som er utroligt korte m�der, passer godt til software-udvikling, ikke s� godt til rapport-skrivning.

Dog har vi kigget p� de enkelte \textbf{practices} og valgt nogle ud som er vurderet til at v�re brugbare.

\paragraph{UP} er valgt helt fra.
Dette skyldes at vi er et meget lille team med en meget \textbf{chaos}-orienteret \textbf{culture}.
Der er derfor ikke brug for den slags strikse metoder og dokumentation, da der er i forvejen god vidensdeling og ingen udskiftninger i projektgruppen.