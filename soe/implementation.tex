\section{Vores metode}
Vi har valgt at tage udgangspunkt i SCRUM, da den ligger godt op ad projektgruppens natur (ift. Boehms stjerne-model).
I dette afsnit vil implementationen af vores udviklingsmetode blive pr�senteret og hvor det er n�dvendigt vil begrundelsen for et valg blive givet.

\subsubsection{Roller}
\begin{description}
\item[Product owner]{Gruppen selv, med fokus p� studieordningen}
\item[Scrum master]{Denne rolle tildeles et nyt gruppemedlem i starten af et sprint (g�r p� rotation).
Opgaver for Scrum masteren best�r i at starte og lede morgen- og sprintm�der, at opdatere backlog og arbejdsplan.}
\item[Team]{En fast gruppe p� 6 mand, alle med evner til at udf�re alle mulige arbejdsopgaver.
Der vil ikke blive foretaget udskiftninger i projektforl�bet.}
\end{description}

\subsubsection{Sprint}
Der vil blive k�rt sprints af l�ngden: 1 uge.
Dette menes som en uge i time-antal, for at tage h�jde for forel�sninger og andet der begr�nser projekt-tiden i en given uge.

\subsubsection{Morgenm�de}
Hver morgen kl. 08.15 (s�fremt der ikke er forel�sning) vil der blive holdt et morgenm�de, hvor der for hver gruppemedlem vil blive givet f�lgende status:
\begin{enumerate}
\item{Hvad lavede du i g�r?}
\item{Hvad skal du lave i dag?}
\item{Er der noget i vejen?}
\end{enumerate}
I forbindelse med dette opdaterer Scrum master arbejdsplanen.

\subsubsection{Sprint planl�gningsm�de}


\subsubsection{Sprint vurdering}