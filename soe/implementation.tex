{ %Der anvendes curly brackets til at lave lokal kommando

\newcommand{\XPhere}{\marginpar{\textbf{\footnotesize XP}}}

\section{Vores metode}\label{ourmethod}
Vi har valgt at tage udgangspunkt i SCRUM (jf. \cref{existing}) og derudover inddrage elementer fra XP til is�r at styrke programmeringsprocessen.
I dette afsnit vil implementationen af vores udviklingsmetode blive pr�senteret og hvor det er n�dvendigt vil begrundelsen for et valg blive givet.
For at kunne identificere de elementer, vi i det f�lgende har udvalgt fra XP, vil de blive markeret med \textbf{XP} i sidens margen.

\subsection{Roller}
SCRUM arbejder med et s�t roller, der typisk tildeles bestemte personer.
I kraft af, at dette projekt er et semesterprojekt laver vi nogle s�rlige udgaver af disse roller.
Herunder beskrives og tildeles de forskellige roller.
\begin{description}
\info{Product owner}
{har ansvaret for at bestemme fokus for projektet.
Product owner skal prioritere opgaver ved planl�gning af et sprint og vurdere om det der produceres kan anvendes eller ej.}
{Eftersom dette semester projekt ikke udvikles til en kunde eller med en bestemt kunde i tankerne, findes der ikke en udenforst�ende til at varetage denne rolle.
Gruppen p�tager sig selv rollen og foretager prioritering/vurdering ud fra studieordningen og projektets problemformulering.}

\info{Scrum master}
{s�rger for at gruppen arbejder produktivt, med udgangspunkt i SCRUM v�rdier.}
{Denne rolle tildeles et nyt gruppemedlem i starten af hvert sprint, s�ledes at alle gruppens medlemmer p� skift p�tager sig rollen.
Gruppen anvender denne model for at alle medlemmer kan fors�ge sig i rollen.
Scrum masterens opgave er at starte og lede morgen- og sprint- og vejlederm�der.
Til morgen- og sprintm�der har scrum master desuden til opgave at opdatere backlog og arbejdsplan.}

\info{Team}
{beskriver det team der arbejder p� et givent projekt og stiller en r�kke krav til teamets medlemmer.}
{Da projektet udvikles af en studiegruppe best�r teamet af en fast gruppe p� 6 mand, alle med evner til at udf�re alle mulige arbejdsopgaver.}
\end{description}

\subsection{Ceremonier}


\subsection{Sprint}
Der vil blive k�rt sprints af l�ngden: 1 uge.
Dette menes som en uge i time-antal, for at tage h�jde for forel�sninger og andet der begr�nser projekt-tiden i en given uge.

\subsection{Morgenm�de}
Hver morgen kl. 08.15 (s�fremt der ikke er forel�sning) vil der blive holdt et morgenm�de, hvor der for hver gruppemedlem vil blive givet f�lgende status:
\begin{enumerate}
\item{Hvad lavede du i g�r?}
\item{Hvad skal du lave i dag?}
\item{Er der noget i vejen?}
\end{enumerate}
I forbindelse med dette opdaterer Scrum master arbejdsplanen.
%Scrummaster stiller sig ved planen og opdaterer den

\subsubsection{Sprint planl�gningsm�de}


\subsubsection{Sprint vurdering}
}
