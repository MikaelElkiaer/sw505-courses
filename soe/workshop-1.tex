% Document class, language and encoding setup
\documentclass[a4paper,12pt,danish]{article}

\usepackage[danish]{babel}
\usepackage[latin1]{inputenc}
\usepackage[T1]{fontenc}
% Fixing the font issue
\usepackage{ae,aecompl}

% Color package
\PassOptionsToPackage{dvipsnames}{xcolor}
	\RequirePackage{xcolor} % [dvipsnames] 
	
% Allows page-links in pdf file
\usepackage[colorlinks=true]{hyperref}

% Allows commands to emit a space "at the end"
\usepackage{xspace}

% Listings setup
\usepackage{listings}

% Graphics setup
\usepackage{graphicx}

% Tikz til spiderweb diagram
\usepackage{tikz}
\usetikzlibrary{shapes}

\newcommand{\D}{5} % number of dimensions (config option)
\newcommand{\U}{5} % number of scale units (config option)

\newdimen\R % maximal diagram radius (config option)
\R=4cm 
\newdimen\L % radius to put dimension labels (config option)
\L=4.7cm

\newcommand{\A}{360/\D} % calculated angle between dimension axes  

\begin{document}

\begin{titlepage}
\newcommand{\HRule}{\rule{\linewidth}{0.5mm}}

\begin{center}

\HRule \\[0.5cm]
\textsc{ \Huge SOE Mini-projekt 1 \\ \Large Udviklingsmetode}\\[0.3cm]

\HRule \\[1cm]

\textsc{\Large SW505E13}

\vfill

{\large Efter�rssemesteret 2013}

\end{center}

\end{titlepage}

\section{Beskrivelse af projekt}
Projektet handler om udvikling af en metode til autonom mapping, vha. en \legoms NXT\textregistered-robot, med kendt position ved brug af en \mskinect til lokalisering.

\section{Den valgte udviklingsmetode}\label{valgtmetode}
I projektperioden har gruppen valgt at arbejde med en metode der tager udgangspunkt i SCRUM, men samtidig benytter elementer fra XP.

Valget er prim�rt truffet ud fra den problemstilling vi som studerende st�r overfor i forbindelse med et semester projekt; alts� at der skal produceres b�de en rapport og k�rende software.
Det er gruppens vurdering at XPs prim�re styrker ligger i programmeringsfasen, men halter i forhold til rapportskrivning.
Dele af XP er derfor tilvalgt for at st�tte op om metoden anvendt til programmering.

I det f�lgende afsnit vil vi beskrive de metoder vi har unders�gt, samt begrunde vores valg af metode.

\subsection{Boehms stjerne-model}
Til at hjælpe med valg af udviklingsmetode er der taget udgangspunkt i Boehms stjerne-model\cite{boehm}.
Modellen beskriver projektgruppens arbejdsform ud fra fem faktorer.
Herunder gennemgås disse faktorer og der gives en vurdering af gruppen i relation til disse.
Gruppens vurderinger er plottet på \cref{boehms:stjerne}.
{
\newcommand{\D}{5} % number of dimensions (config option)
\newcommand{\U}{5} % number of scale units (config option)
\newcommand{\UU}{50}

\newdimen\R % maximal diagram radius (config option)
\R=3.5cm 
\newdimen\L % radius to put dimension labels (config option)
\L=3.7cm

\newcommand{\A}{360/\D} % calculated angle between dimension axes  

\begin{figure}[h]
\centering
\begin{tikzpicture}[scale=1]
  \path (0:0cm) coordinate (O); % define coordinate for origin

  % draw the spiderweb
  \foreach \X in {1,...,\D}{
    \draw (\X*\A+90:0) -- (\X*\A+90:\L);
  }

  \foreach \Y in {0,...,\U}{
    \foreach \X in {1,...,\D}{
      \path (\X*\A+90:\Y*\R/\U) coordinate (D\X-\Y);
      \fill (D\X-\Y) circle (1pt);
    }
%    \draw [opacity=0.3] (90:\Y*\R/\U) \foreach \X in {1,...,\D}{
%        -- (\X*\A+90:\Y*\R/\U)
%    } -- cycle;
  }
  \foreach \X in {1,...,\D}{
    \path (\X*\A+90:5.7*\R/\U) coordinate (D\X-T);
  }
  \foreach \Y in {0,...,\UU}{
    \foreach \X in {1,...,\D}{
      \path (\X*\A+90:\Y*\R/\UU) coordinate (D\X--\Y);
    }
  }
  
  \path (D5-T) node (L1) {\scriptsize Personnel};
  \draw (D5-5) node[left] {\tiny (1A/1B) 100};
  \draw (D5-5) node[right] {\tiny 0 (2/3)};
  \draw (D5-4) node[left] {\tiny 75};
  \draw (D5-4) node[right] {\tiny 25};
  \draw (D5-3) node[left] {\tiny 50};
  \draw (D5-3) node[right] {\tiny 50};
  \draw (D5-2) node[left] {\tiny 25};
  \draw (D5-2) node[right] {\tiny 75};
  \draw (D5-1) node[left] {\tiny 0};
  \draw (D5-1) node[right] {\tiny 100};
  
  \path (D4-T) node (L2) {\scriptsize Dynamism};
  \draw (D4-5) node[below] {\tiny 1};
  \draw (D4-4) node[below] {\tiny 5};
  \draw (D4-3) node[below] {\tiny 10};
  \draw (D4-2) node[below] {\tiny 30};
  \draw (D4-1) node[below] {\tiny 50};
  
  \path (D3-T) node (L3) {\scriptsize Culture};
  \draw (D3-5) node[left] {\tiny 10};
  \draw (D3-4) node[left] {\tiny 30};
  \draw (D3-3) node[left] {\tiny 50};
  \draw (D3-2) node[left] {\tiny 70};
  \draw (D3-1) node[left] {\tiny 90};
  
  \path (D2-T) node (L4) {\scriptsize Size};
  \draw (D2-5) node[right] {\tiny 300};
  \draw (D2-4) node[right] {\tiny 100};
  \draw (D2-3) node[right] {\tiny 30};
  \draw (D2-2) node[right] {\tiny 10};
  \draw (D2-1) node[right] {\tiny 3};
  
  \path (D1-T) node (L5) {\scriptsize Criticality};
  \draw (D1-5) node[below, align=center, font=\tiny] {Many\\Lives};
  \draw (D1-4) node[above, align=center, font=\tiny] {Single\\Life};
  \draw (D1-3) node[below, align=center, font=\tiny] {Essential\\Funds};
  \draw (D1-2) node[above, align=center, font=\tiny] {Discretionary\\Funds};
  \draw (D1-1) node[below, align=center, font=\tiny] {Comfort};
    
  \draw [color=blue,line width=1.5pt,opacity=0.5]
    (D1--10) --
    (D2--14) --
    (D3--15) --
    (D4--40) --
    (D5--10) -- cycle;

\end{tikzpicture}
\caption{Boehms stjerne-model}
\label{boehms:stjerne}
\end{figure}
}

\begin{description}
\info{Personnel}{beskriver projektgruppens medlemmers ''niveau'' i relation til programmering.
Denne inddeling sker efter Cockburns skala \cite[Tabel 2]{boehm}, der beskriver niveau i relation til det der skal produceres.}
{Vi har i projektgruppen vurderet at alle gruppens medlemmer er på niveau 2 eller 3.}

\info{Dynamism}{angiver i hvilken grad det forventes at der opstår ændringer i et projekts kravspecifikation.
Det fortæller altså i hvilken grad projektgruppen skal være i stand til at tilpasse sig forandringer i krav.}
{Da vi arbejder problembaseret, tages der udgangspunkt i nogle simple og konkrete krav, der i løbet af projektperioden kun vil blive yderligere præciseret.}

\info{Culture}{er en vurdering af hvorvidt en projektgruppes medlemmer trives bedst med kaos eller orden.
På skalaen angives ''hvor meget'' kaos gruppen trives med.}
{Projektgruppens medlemmer har hver især vurderet hvor de ligger på skalaen.
Gennemsnittet af disse vurderinger er på 69,2\%.}

\info{Size}{udtaler sig udelukkende om størrelsen på den gruppe der skal arbejde på et projekt.}
{Projektgruppen består af 6 personer.}

\info{Criticality}{beskriver ''hvor vigtigt'' det er at det endelige produkt er fungerende og lever op til alle krav.
Skalaen angiver denne vigtighed i form af hvilken effekt det har hvis dette ikke er tilfældet.}
{Da den robot, som er projektets mål, ikke skal sættes i produktion, vil den aldrig have højere betydning end \textbf{Comfort} på skalaen.
Det er naturligvis et mål for projektgruppen at have en fungerende robot ved semestrets afslutning.
Men da den tekniske forståelse, som er et biprodukt af det udviklede, ligeledes udbygges, vurderer vi det ikke som kritisk hvis robotten ikke er fuldt fungerende.}
\end{description}

Ud fra modellen (\cref{boehms:stjerne}) ses det at projektgruppen med fordel kan vælge en agil metode til projektarbejdet, da plottet ligger nærmest midten for næsten alle punkter.
Det eneste punkt der ikke direkte anbefaler en agil tilgang er \emph{Dynamism}.
Dog påpeger Boehm og Turner \cite[side 2]{boehm} at dette ikke er et problem:
\quoter{Boehm og Turner}{``For Dynamism, agile methods are at home with 
both high and low rates of change, but plan-driven 
methods prefer low rates of change.''} 
Altså har få krav ikke betydning for valget mellem en agil og en traditionel metode.
Vi kan nu skifte fokus til valg af en specifik agil metode.
I det følgende afsnit gives en kort beskrivelse af udvalgte agile metoder.
\subsection{Eksisterende metoder}\label{existing}
Til at finde en passende udviklingsmetode til projektgruppen er der kigget p� f�lgende eksisterende agile metoder: SCRUM, XP og UP.\cite{larman}
Der er kun kigget netop disse metoder, da de er blevet pr�senteret i \emph{Software Engineering} kurset som v�rende meget udbredte metoder.
Vi kan derfor med fordel v�lge en af disse metoder til vort studieprojekt.


\paragraph{XP} er en agil metode der tager udgangspunkt i at de udviklere der er i gruppen har et h�jt fagligt niveau.
Med det udgangspunkt frav�lges en r�kke elementer fra plandrevet udvikling der ikke giver kunden direkte v�rdi.
P� et semesterprojekt har vi ikke en egentlig kunde at udvikle til, hvilket p�virker afvejningen af hvad der skaber v�rdi.
Samtidig er vi i en l�ringsproces hvor den bagved liggende teori er af stor relevans, og skal dokumenteres i form af en rapport.
Derfor har vi fravalgt XP som overordnet metode, men vil i \cref{ourmethod} ``\nameref{ourmethod}'' beskrive nogle af de elementer vi inkluderer til programmering.

\paragraph{UP} er en ligeledes en agil metode til software udvikling.
UP er dog baseret p� st�rre teams med udskiftning og sikrer via forskellige typer dokumentation at denne udskiftning er mulig.
Eftersom det ``team'' vi arbejder i, i form af projektgruppen, kun er p� seks mand og derfor ikke s�rlig stort (jf. \cref{boehms:stjerne}) og helt uden udskiftning har vi helt fravalgt UP.
Vi ser desuden ogs� p� Boehms model at der er en h�j grad af kaos hvilket strider mod den rigide dokumentation i UP.

\paragraph{SCRUM} stemmer godt overens med projektet, da den er meget fleksibel og nem at tilpasse.


\section{Vores metode}
Vi har valgt at tage udgangspunkt i SCRUM, da den ligger godt op ad projektgruppens natur (ift. Boehms stjerne-model).
I dette afsnit vil implementationen af vores udviklingsmetode blive pr�senteret og hvor det er n�dvendigt vil begrundelsen for et valg blive givet.

\subsubsection{Roller}
\begin{description}
\item[Product owner]{Gruppen selv, med fokus p� studieordningen}
\item[Scrum master]{Denne rolle tildeles et nyt gruppemedlem i starten af et sprint (g�r p� rotation).
Opgaver for Scrum masteren best�r i at starte og lede morgen- og sprintm�der, at opdatere backlog og arbejdsplan.}
\item[Team]{En fast gruppe p� 6 mand, alle med evner til at udf�re alle mulige arbejdsopgaver.
Der vil ikke blive foretaget udskiftninger i projektforl�bet.}
\end{description}

\subsubsection{Sprint}
Der vil blive k�rt sprints af l�ngden: 1 uge.
Dette menes som en uge i time-antal, for at tage h�jde for forel�sninger og andet der begr�nser projekt-tiden i en given uge.

\subsubsection{Morgenm�de}
Hver morgen kl. 08.15 (s�fremt der ikke er forel�sning) vil der blive holdt et morgenm�de, hvor der for hver gruppemedlem vil blive givet f�lgende status:
\begin{enumerate}
\item{Hvad lavede du i g�r?}
\item{Hvad skal du lave i dag?}
\item{Er der noget i vejen?}
\end{enumerate}
I forbindelse med dette opdaterer Scrum master arbejdsplanen.

\subsubsection{Sprint planl�gningsm�de}


\subsubsection{Sprint vurdering}


\begin{thebibliography}{9}

\bibitem{boehm}
  Boehm, B.; Turner, R., ''Observations on balancing discipline and agility'', Agile Development Conference, 2003.
  ADC 2003. Proceedings of the , vol., no., pp.32,39, 25-28 June 2003
  
\bibitem{larman}
Larman, Craig., ''Agile and Iterative Development: A Manager's Guide''.
Addison-Wesley, 2004.
\end{thebibliography}

\end{document}