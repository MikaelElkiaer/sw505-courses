% !TeX spellcheck = da_DK
\subsection{Eksisterende metoder}\label{existing}
Til at finde en passende udviklingsmetode til projektgruppen er der kigget på følgende eksisterende agile metoder: SCRUM, XP og UP.\cite{larman}
Der er kun kigget netop disse metoder, da de er blevet præsenteret i \emph{Software Engineering} kurset som værende meget udbredte metoder.
Vi kan derfor med fordel vælge en af disse metoder til vort studieprojekt.


\paragraph{XP} er en agil metode der tager udgangspunkt i at de udviklere der er i gruppen har et højt fagligt niveau.
Med det udgangspunkt fravælges en række elementer fra plandrevet udvikling der ikke giver kunden direkte værdi.
På et semesterprojekt har vi ikke en egentlig kunde at udvikle til, hvilket påvirker afvejningen af hvad der skaber værdi.
Samtidig er vi i en læringsproces hvor den bagved liggende teori er af stor relevans, og skal dokumenteres i form af en rapport.
Derfor har vi fravalgt XP som overordnet metode, men vil i \cref{ourmethod} ``\nameref{ourmethod}'' beskrive nogle af de elementer vi inkluderer til programmering.

\paragraph{UP} er en ligeledes en agil metode til software udvikling.
UP er dog baseret på større teams med udskiftning, og sikrer via forskellige typer dokumentation at denne udskiftning er mulig.
Eftersom det ``team'' vi arbejder i er på seks mand, og derfor ikke er særlig stort (i form af projektgruppen, jf. \cref{boehms:stjerne}) og helt uden udskiftning, har vi helt fravalgt UP.
Vi ser desuden også på Boehms model at der er en høj grad af kaos, hvilket strider mod at der er meget dokumentation i UP.

\paragraph{SCRUM} er en agil udviklingsmetode der består af små selvstyrede hold med daglige evalueringer af fremskridt. 
I scrum er produktet i fokus, og kunden inddrages ved hver iteration for at sikre at projektet er på vej i den rigtige retning. Scrum er meget fleksibel og nem at tilpasse, hvilket gør den let at implementere i et semesterprojekt. 
Vi har derfor valgt at bruge scrum som udgangspunkt for vores udviklingsmetode i dette projekt.
