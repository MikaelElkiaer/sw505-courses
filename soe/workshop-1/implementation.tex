{ %Der anvendes curly brackets til at lave lokal kommando

\newcommand{\XPhere}{\marginpar{\textbf{\footnotesize XP}}}

\section{Vores metode}\label{ourmethod}
Vi har valgt at tage udgangspunkt i SCRUM (jf. \cref{existing}) og derudover inddrage elementer fra XP til især at styrke programmeringsprocessen.
I dette afsnit vil implementationen af vores udviklingsmetode blive præsenteret, og hvor det er nødvendigt vil begrundelsen for et valg blive givet.

\subsection{Roller}
SCRUM arbejder med et sæt roller, der typisk tildeles bestemte personer.
I kraft af at dette projekt er et semesterprojekt, laver vi nogle særlige udgaver af disse roller.
Herunder beskrives og tildeles de forskellige roller.
\begin{description}
\info{Product owner}
{har ansvaret for at bestemme fokus for projektet.
Product owner skal prioritere opgaver ved planlægning af et sprint og vurdere om det der produceres kan anvendes eller ej.}
{Eftersom produktet af dette semester projekt ikke udvikles til en kunde eller med en bestemt kunde i tankerne, findes der ikke en udenforstående til at varetage denne rolle.
Gruppen påtager sig selv rollen og foretager prioritering/vurdering ud fra studieordningen og projektets problemformulering.}

\info{Scrum master}
{sørger for at gruppen arbejder produktivt, med udgangspunkt i SCRUM værdier.}
{Denne rolle tildeles et nyt gruppemedlem i starten af hvert sprint, således at alle gruppens medlemmer på skift påtager sig rollen.
Gruppen anvender denne model for at alle medlemmer kan forsøge sig i rollen.


Scrum masterens opgave er at starte og dirigere morgen-, sprint- og vejledermøder.}

\info{Team}
{beskriver det team der arbejder på et givent projekt og stiller en række krav til teamets medlemmer.}
{Da projektet udvikles af en studiegruppe består teamet af en fast gruppe på 6 mand; alle med evner til at udføre alle mulige arbejdsopgaver.}
\end{description}

\subsection{Ceremonier}

Der vil blive kørt sprints af længden 1 uge.
Dette menes som en uge i time-antal for at tage højde for forelæsninger og andet der begrænser projekt-tiden i en given uge.
Timeantallet for et sprint er 36 timer og er bestemt ud fra at vi arbejder 8 timer om dagen undtagen fredag hvor der regnes med 4 timer pga. \nameref{friday_meeting} som specificeret i \cref{friday_meeting}.
I løbet af et sprint vil følgende ceremonier være en del af processen.

\subsubsection{Estimeringsmøde}
Først laves stories der samles i en backlog. 
Disse stories skal beskrive de krav projektet har. 
Stories bliver lavet ud fra projektets problemformulering og studieordningen.
Disse stories prioriteres efter hvor vigtige de er for slutproduktet.

Der spilles planning poker om hver enkelt opgave.
Hvis en story er for stor bliver den opdelt i mindre opgaver.
Der spilles planning poker indtil alle opgaver vurderes til at kunne løses inden for 2 dage som er det scrum\cite{larman} anbefaler.

\subsubsection{Sprint planlægningsmøde}
Sprint planlægningsmødet foregår inden hvert sprint. 
Der planlægges hvilke stories der tages med til det næste sprint. 
Opgaver diskuteres så det er klart hvilke krav opgaven indeholder, samt hvornår opgaven er løst.
\todo[inline]{Dette skal muligvis revideres når vi har prøvet det et par gange. Se \cite{checklist} s.6 - procedure}

\subsubsection{Morgenmøde}
Hver morgen kl. 08.15 (såfremt der ikke er forelæsning) vil der blive holdt et morgenmøde, hvor der for hvert gruppemedlem vil blive givet følgende status:
\begin{itemize}
\item{Er der nogen der skal gå før tid i dag?}
\end{itemize}
Derefter bliver hver enkelt person spurgt af Scrum Masteren:
\begin{enumerate}
\item{Hvad lavede du i går?}
\item{Hvad skal du lave i dag?}
\item{Er der noget i vejen?}
\end{enumerate}
I forbindelse med dette opdaterer hvert gruppemedlem scrumboardet.

\subsubsection{Sprint vurdering}
Sprint vurdering indledes med at ridse op hvad det netop afsluttede sprint indeholdte. 
Hvert gruppemedlem fremlægger derefter kort de resultater de har opnået i sprintet.
\todo[inline]{Dette skal muligvis revideres når vi har prøvet det et par gange. Se \cite{checklist} s.20 - procedure}

\subsubsection{Fredagsmøde}\label{friday_meeting}
Hver fredag holdes et fredagsmøde, hvis formål er at identificere og udbedre eventuelle sociale problemer i gruppen.
Fredagsmødet styres af scrum masteren som følger en specifik dagsorden, der blandt andet indeholder spørgsmål som: ``Hvordan går det'' og ``konsruktiv kritik til andre gruppemedlemmer. 
Fredagsmødet har samtidig til formål at være en afslappet og social afslutning på ugen. \todo[inline]{skal denne inkluderes? jeg synes det bliver for meget}
\todo[inline]{synes jeg den skal -- man kunne endvidere gå så langt også at nævne Curve. Mvh. Thilemann.}
\todo[inline]{enig med thiele - Bronx}

\subsection{Praksis fra XP}\label{method:xp:practice}
Som nævnt i \cref{ourmethod} har vi udvalgt nogle praksisser fra XP, som vi vil benytte i forbindelse med programmeringsopgaver.
Disse praksisser vil her blive beskrevet kort.

\paragraph{Simple design}
Ved løsning af en opgave vælges den mest simple løsning, så der fokuseres på at få løst kerneproblemet.

\paragraph{Refactoring}
Kodebasen vil løbende blive refaktoriseret for at eliminere duplikeret kode samt for at forsimple og/eller forbedre det eksisterende.
%Der vil løbende blive udført refaktoriset kode så dublikeret kode fjernes og koden generelt bliver forsimplet eller forbedret.

\paragraph{Pair programming}
Gruppen vil forsøge at bruge denne praksis i situationer hvor der er få programmeringsopgaver, men ikke som en generel praksis.
\todo[inline]{måske også hvorfor?}

\paragraph{Collective ownership}
Det er gruppens filosofi at alle ejer både kode og rapport, hvorfor alle kan lave ændringer hvis det vurderes nødvendigt.

\paragraph{Sustainable pace}
Sprints er bestemt til at vare en uge for at holde et konsistent tempo dvs. at arbejde overtid ikke er tilladt.
Dette gør det også lettere at planlægge sprints.

\paragraph{Code standards}
For at ensarte koden har gruppen fastsat en række kodestandarder der skal overholdes.
\todo[inline]{Skal der en begrundelse til hver?}