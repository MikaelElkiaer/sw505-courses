\section{Implementations strategi}
For at kunne implementere design patterns i gruppens projektarbejde, vil vi søge at øge gruppemedlemmernes kendskab til forskellige design patterns.
Dette skal være med til at sikre, at alle gruppens medlemmer har det fornødne kendskab til design patterns.
Som værktøj til at få denne viden afholder gruppen en design pattern workshop.
Målet med denne er at foretage vidensdeling om forskellige design patterns.

For at kunne gøre dette, er det nødvendigt først at fastslå hvilke design patterns gruppen vælger at fokusere på.
Til dette har gruppen valgt at tage udgangspunkt i en oversigt over design patterns (bilag \ref{appendix:patterns}) der blev refereret i Software Engineering kurset.

Gruppens medlemmer vil fordele de forskellige design patterns indbyrdes, undersøge dem og efterfølgende præsentere dem for hinanden.
Igennem præsentationen vil der være mulighed for at spørge ind til detaljerne i det enkelte design pattern.

En præsentation af et design pattern skal indeholde følgende punkter:
\begin{description}
\item[Formål] -- En kort beskrivelse af mønstrets overordnede funktion.
\item[Motivation] -- Eksempler på hvilke problemstillinger mønstret kan løse.
\item[UML] -- Præsentation af et klassediagram der beskriver strukturen i det enkelte mønster.
\item[Eksempel] -- Implementation (i C\#) af mønstret, der viser hvordan det kan løse problemstillinger nævnt i \emph{Motivation}.
\end{description}

For bedre at sikre en god forståelse af alle design patterns gennemgåes de alle af to gruppemedlemmer.
I listen (bilag \ref{appendix:patterns}) optræder 23 design patterns.
Da gruppen består af 6 mand, tildeles \textit{alle} således 8 design patterns.
Det er tilfældigt udvalgt hvilke gruppemedlemmer der undersøger hvilke patterns.
Hvert design pattern præsenteres af \'et gruppemedlem.
Hvem der præsenterer er ligeledes tilfældigt udvalgt.

Tidsplanen for workshop er som følger:
\begin{description}
\item [8:15-11:30] Undersøgelse af design patterns.
\item [11:30-12:30] Frokostpause (i henhold til gruppekontrakten).
\item [12:30-16:15] Præsentation af design patterns.
\end{description}

Med udgangspunkt i den valgte metode (\cref{method:xp:practice}), mere specifikt anvendelsen af refaktorering, vil gruppen sikre brugen af den viden der opnås på workshoppen ved at fokusere på design patterns når kode refaktoreres.
På den måde kan design patterns anvendes retrospektivt (og i mindre grad som et planlægnings-værktøj), hvilket er i tråd med den agile metode vi i gruppen anvender.