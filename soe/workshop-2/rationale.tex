\section{Rationale for valg}
I dette afsnit vil der være en begrundelse af de valgte værktøjer.
Værktøjerne er valgt på baggrund af SWOT analysen præsenteret i \cref{swot:analyse}.

\subsection{Design Patterns}\label{workshop2:designpatterns}
Design patterns er valgt pga. følgende punkter i SWOT analysen:
\begin{enumerate}
\item Forskelligt niveau/kvalitet af kode (\cref{swot:niveau_af_kode})
\item Kode retningslinjer - ''bad smells'' (\cref{swot:bad_smells})
\end{enumerate}

Som man kan se på de ovenståeden punkter skal design patterns være med til at hæve niveauet for koden og ensrette forståelsen af den.
Det skal være med til at fjerne nogle af de ''bad smells'' der kan forekomme.
Hvis alle bruger design patterns, vil det blive nemmere at læse, og derefter også reaktorfere koden for andre.

\subsection{Risk Management}
Risk Management er valgt pga. følgende punket i SWOT analysen:

\begin{enumerate}
\item Manlgende viden omkring MI (\cref{swot:manglende_viden_mi})
\item Usikkerhed pga. løbende optagelse af viden (\cref{swot:optagelse_af_viden})
\item Misforståede læringsmål (\cref{swot:laeringsmaal})
\item Vejleder skifter mening undervejs (\cref{swot:vejleder})
\end{enumerate}

Som det kan ses på de ovenstående punkter er det en meget god ide at lave risk management med de risikoer hvert punkt indeholder.
Men da det tager lang tid at lave risk management, og der er begrænset tid til projektet, er det fravalgt da det vurderes at afkastet ikke står mål med indsatsen det vil kræve.