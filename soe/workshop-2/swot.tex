\section{SWOT analyse}\label{swot:analyse}
I dette kapitel er projektet analyseret i forhold til dets styrker, svagheder, muligheder og trusler (SWOT analyse), for at give et overblik over projektes indre formåen (styrker og svagheder), men også for at belyse de elementer som kan påvirke projektet positivt og negativt (muligheder og trusler).

Værktøjer der allerede er inddraget i projektet er ikke beskrevet, på trods af at dette kapitel beskriver værktøjer og teknikker.
Dette valg er taget på baggrund af ønsket om at undersøge nye værktøjer som kan styrke projektarbejdet yderligere.

\subsection{Styrker \textnormal{(\textbf{S}trengths)}}

\paragraph{Mere end ét grupperum}
Det ses som en fordel at gruppen er blevet tildelt et ekstra grupperum, da det giver bedre mulighed for at teste robotten i et testmiljø.
Det fjerner desuden også støj fra det ''rigtige'' grupperum, således der kan arbejdes mere koncentreret.

\paragraph{Ikke SCRUM eller XP}
Softwareudviklingsmetoden der benyttes i projektet er hverken SCRUM eller XP, men derimod et mix af hvad gruppen anser som det bedste fra begge verdener \cref{ourmethod}.
Dette har vist sig at give et godt fokus, da gruppen hele tiden ved hvad hvert gruppemedlem arbejder på, samt eventuelle problemer der er opstået.
Metoden har vist sig at virke godt i praksis, på trods af at alle metoder endnu ikke er afprøvet (f.eks. par-programmering) -- hvilket tolkes af gruppen som metoden endnu ikke har vist sit fulde potentiale.

\paragraph{Egen udviklingsmetode}
Det er en styrke at gruppen har dets egen udviklingsmetode tilpasset til gruppens behov og projekt (som beskrevet i \cref{udviklingsmetode}).

\paragraph{Matematisk baggrund}
Det anses som en fordel at alle i gruppen besidder en form for matematisk baggrund (gymnasie, HTX osv.).

\paragraph{Veldefinerede minimumsmål for projektet}
Gruppen har fundet god enighed omkring målene for projektet; disse er endvidere defineret i form af diverse ''stories'' som beskriver en overordnet funktionalitet af det endelige produkt.
Dette koblet sammen med det initierende problem giver et godt samlet udtryk af hvad der skal arbejdes hen imod.

\paragraph{Godt samarbejde mellem gruppens medlemmer}
Der er i gruppen kun ganske få konflikter (navnligt når der diskuteres essentielle emner i forhold til fremtidig retning af projektet).
Der forefindes også en gruppekontrakt der respekteres af alle gruppens medlemmer, hvilket fremmer konfliktløsning.

\paragraph{God arbejdsmoral}
Der er i gruppen stor enighed om hvornår der skal holdes pause eller ej.
Derudover arbejdes der efter bestemte arbejdstider, som overholdes af alle gruppens medlemmer.
Alt arbejde foregår også i grupperummene, hvilket gør det nemt at stille spørgsmål til andre gruppemedlemmer.

\paragraph{Få sygedage}
Der er kun afholdt ganske få sygedage, hvilket fremmer gruppearbejdet.

\subsection{Svagheder \textnormal{(\textbf{W}eaknesses)}}\label{swot:weaknesses}

\paragraph{Manglende viden omkring MI}\label{swot:manglende_viden_mi}
Dette anses som værende en af gruppens største svagheder; dog er det forventeligt, da ingen i gruppen før har arbejdet med Maskin Intelligens (eller sandsynligheder) og at kurses indhold præsenteres løbende.

\paragraph{Usikkerhed pga. løbende optagelse af viden}\label{swot:optagelse_af_viden}
Hænger delvist sammen med ovenstående, da ikke alt indhold i kurserne præsenteres i den rækkefølge der nødvendigvis er mest relevant i forbindelse med projektets fremgang.

\paragraph{Forskelligt niveau/kvalitet af kode}\label{swot:niveau_af_kode}
På trods af at alle gruppens medlemmer skriver kode af fornuftig kvalitet, så er der stadig forskel på medlemmers erfaring og hvilken type sprogkonstruktioner der benyttes.
Dette kan gøre noget kode svært læseligt for nogle og kan dermed også få betydning for kvaliteten når al kode samles.

\paragraph{Kode retningslinjer - bad smells}\label{swot:bad_smells}
Gruppekontrakten indeholder diverse retningslinjer for kode; det er dog ikke altid muligt at fjerne alle ''bad smells'', da alle koder forskelligt.
Gruppekontrakten skal fjerne de største uenigheder.
Mindre forskelle kan tolereres -- sålænge koden er velskrevet og læselig.

\subsection{Muligheder \textnormal{(\textbf{O}pportunities)}}

\paragraph{Hjælpsom og engageret vejleder}
Dette punkt anses som værende af essentiel kvalitet for hele projektets fremgang. 
Da gruppen har arbejdet sammen om flere projekter og derfor også med flere vejledere, findes der store muligheder i at nuværende vejleder engagerer sig 100 procent i projektarbejdet.
Dette er med til at skubbe projektet i den rigtige retning og gør det ligeledes muligt at holde fokus på studieordningens mål.

\paragraph{Kontakt til andre grupper}
Det findes værdi i muligheden for at tage kontakt til andre projektgrupper som arbejder med en lignende problemstilling som dette projekt.
Støder vi på et problem og konkluderer at det ikke kan løses med den viden vi har tilegnet os, kan de andre grupper med fordel besøges for at få idéer til en videre løsning.

\paragraph{Adgang til udstyr}
Det anses af gruppen at være en stor mulighed at der stilles diverse udstyr tilrådighed af universitetet.
Desuden er det også nemt og hurtigt at fremskaffe nyt/alternativt udstyr, hvis der pludselig opstår mangler.

\paragraph{Ny viden}\label{swot:ny_viden}
Der findes store muligheder i at tilegne sig ny og relevant viden især via kurser, men også andet eksternt materiale.
Således vil bøger, artikler, internettet og generelt som mange kilder som muligt, blive inddraget i informationssøgning (og delt i gruppen).
Dette kan sætte projektet i et nyt perspektiv og afhjælpe fremtidigt arbejde.

\paragraph{Git Issues}
Da både vores rapport(er) og kode for projektet er versionsstyret (via \url{http://github.com}), er der mulighed for at benytte en funktionalitet de kalder for ''Git Issues''.
Dette er en avanceret form for todo-notes som følger hvert enkelt commit.
Det gør det muligt at oprette et problem (såkaldt \textit{issue}), hvis man finder fejl i kode.
Dette kan højne kvaliteten af projektet, da det giver en overisgt over de problemer der skulle være.

\subsection{Trusler \textnormal{(\textbf{T}hreats)}}

\paragraph{Misforståede læringsmål}\label{swot:laeringsmaal}
Der er i studieordningen fundet et afsnit der siger at man i dette projekt skal kunne fremvise et kørende indlejret system.
Dette har givet anledning til forvirring, da der andetsteds står at der blot skal inddrages relevant teori fra de forskellige kurser, hvilket i vores tilfælde er MI (maskin intelligens).
Da der ikke har været fokus på at bygge et indlejret system, men derimod et intelligent system, har dette aspekt manglet i vores løsning.
Det er efterfølgende blevet besluttet at bygge en mindre del af løsningen som et indlejret system, men dette aspekt anses stadig som en trussel, da der endnu ikke forefindes en endelig afklaring.

\paragraph{Vejleder skifter mening undervejs}\label{swot:vejleder}
På trods af at vi er meget tilfredse med vejleder samarbejdet, anses det stadig som en trussel hvis han skulle skifte mening undervejs.
Dette kan betyde at væsentlige mængder arbejde er spildt, hvilket i sidste ende kan betyde at vores tidsplan ikke holder.

