\section{SWOT analyse}
I denne sektion er projektet analyseret i forhold til dets styrker, svagheder, muligheder og trusler (SWOT analyse), for at give et overblik over projektes indre form�en (styrker og svagheder), men ogs� for at belyse de elementer som kan p�virke projektet positivt og negativt (muligheder og trusler).

\subsection{Styrker \textnormal{(\textbf{S}trengths)}}
\begin{itemize}
\item To lokaler
\item not XScrum
\item Matematisk baggrund
\item Veldefinerede minimumsm�l for projektet
\item Godt samarbejde mellem gruppens medlemmer
\item M�des hver dag og alt arbejde foreg�r i grupperum
\item God arbejdsmoral
\item F� sygedage
\end{itemize}

\subsection{Svagheder \textnormal{(\textbf{W}eaknesses)}}
\begin{itemize}
\item Manglende viden omkring MI
\item Forskelligt niveau/kvalitet af kode
\item -Bad smells
\item Usikkerhed pga. l�bende optagelse af viden
\end{itemize}

\subsection{Muligheder \textnormal{(\textbf{O}pportunities)}}
\begin{itemize}
\item Hj�lpsom og engageret vejleder
\item Tilg�ngelige teknologier til configuration Management (f.eks. CruiseControl.net) 
\item Kontakt med andre grupper med lignende projekt
\item Adgang til: Kinect, Lego mindstorms, 
\item Ny viden fra kurser og eksternt materiale.
\item Git issues
\end{itemize}

\subsection{Trusler \textnormal{(\textbf{T}hreats)}}
\begin{itemize}
\item Misforst�ede l�ringsm�l
\item Vejleder skifter mening undervejs
\end{itemize}