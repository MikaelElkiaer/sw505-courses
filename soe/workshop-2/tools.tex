\section{Udvalgte v�rkt�jer og teknikker}
Efter granskningen af SWOT analysen, som beskrevet i \cref{swot_analyse}, er det besluttet at vi vil fokusere p� at arbejde med design patterns, da disse kan tilf�je h�j v�rdi i forhold til de problemer der er fundet med SWOT analysen.
Da gruppen i forvejen kender til mange forskellige typer v�rkt�jer og allerede har inddraget mange i projektarbejdet (elementer fra SCRUM, Xtreme Programming, Git versionsstyring m.m.), ses design patterns som en god udfordring, da det er et element som ikke f�r har v�ret inddraget i udviklingsforl�bet.
F�lgende afsnit (\cref{design_patterns_beskrivelse}) fokuserer s�ledes p� at beskrive hvad design patterns er.

\subsection{Design Patterns}\label{design_patterns_beskrivelse}
Som programm�r er man alle forskellige; som ogs� beskrevet i SWOT analysen er det n�dvendigt at v�re opm�rksom p� f.eks. ''bad smells'' og andre forskelligheder (personlige pr�ferencer) i forhold til hvordan der skrives kode.
Design patterns d�kker s�ledes over forskellige typer skabeloner som skal l�se generelle problemer der opst�r ved softwareudvikling, under foruds�tning af en given kontekst.

En s�dan kontekst kan v�re mange forskellige ting; for eksempel:

\begin{itemize}
\item Et system som arbejder med �n instans af en given klasse (singleton pattern).
\item En teksteditor der �nsker undo/redo funktionalitet (memento/command pattern)
\end{itemize}

Tager vi udgangspunkt i memento, s� giver det ikke en l�sning (kode) til hvordan man implementerer undo/redo funktionalitet.
Meningen er i stedet at tilf�je en afpr�vet (og kendt) struktur, s� man ikke ender med en \textit{adhoc}-l�sning af problemet.
Denne struktur tilf�jer is�r l�sbarhed og g�r koden nemmere at forst�, under foruds�tning af at man kender det p�g�ldende pattern.