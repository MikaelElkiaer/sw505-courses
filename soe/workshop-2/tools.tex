\section{Udvalgte værktøjer og teknikker}
Efter granskningen af SWOT analysen, som beskrevet i \cref{swot:analyse}, er det besluttet at vi vil fokusere på at arbejde med design patterns, da disse kan tilføje høj værdi i forhold til de problemer der er fundet med SWOT analysen.
Da gruppen i forvejen kender til mange forskellige typer værktøjer og allerede har inddraget mange i projektarbejdet (elementer fra SCRUM, Xtreme Programming, Git versionsstyring m.m.), ses design patterns som en god udfordring, da det er et element som ikke før har været inddraget i udviklingsforløbet.
Følgende afsnit (\cref{design_patterns_beskrivelse}) fokuserer således på at beskrive hvad design patterns er.

\subsection{Design Patterns}\label{design_patterns_beskrivelse}
Som programmør er man alle forskellige; som også beskrevet i SWOT analysen er det nødvendigt at være opmærksom på f.eks. ''bad smells'' og andre forskelligheder (personlige præferencer) i forhold til hvordan der skrives kode.
Design patterns dækker således over forskellige typer skabeloner som skal løse generelle problemer der opstår ved softwareudvikling, under forudsætning af en given kontekst.

En sådan kontekst kan være mange forskellige ting; for eksempel:

\begin{itemize}
\item Et system som arbejder med én instans af en given klasse (singleton pattern).
\item En teksteditor der ønsker undo/redo funktionalitet (memento/command pattern)
\end{itemize}

Tager vi udgangspunkt i memento, så giver det ikke en løsning (kode) til hvordan man implementerer undo/redo funktionalitet.
Meningen er i stedet at tilføje en afprøvet (og kendt) struktur, så man ikke ender med en \textit{ad hoc}-struktur af problemet.
Denne struktur tilføjer især læsbarhed og gør koden nemmere at forstå, under forudsætning af at man kender det pågældende pattern.