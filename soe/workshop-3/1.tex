\section{Opsummering}\label{workshop3:opsummering}
I gruppens daglige arbejde har introduktionen af XP/Scrum været den væsentligste ændring.
Den anvendte metode (som beskrevet i \cref{ourmethod}) har i høj grad medvirket til, at gruppen har oplevet en mere struktureret arbejdsgang.
Som biprodukt af dette har gruppens medlemmer fået større overblik over kommende, samt igangværende, opgaver.
I de følgende afsnit opsummeres på gruppens valgte arbejdsmetode.
Herefter (\cref{konsekvenser} og \cref{workshop3:fremtidig_brug}) diskuteres konsekvenserne af de indførte metoder.

\paragraph{Fredagsmøder}
Ved gruppens fredagsmøder (se \cref{friday_meeting}) har gruppen haft mulighed for løbende at evaluere de metoder der har været taget i brug.
På denne måde er metoder blevet evalueret løbende af alle gruppens medlemmer, såfremt der har været kommentarer til metoderne.
\Cref{workshop3:fredag} giver en yderligere beskrivelse af fredagsmøderne.

\subsection{Brugen af sprints}
Som en del af en agil arbejdsgang var gruppen nødt til at vurdere en passende længde af et sprint.
Længden af sprints blev her vurderet til længden af en fuld arbejdsuge, altså 36 arbejdstimer.
Det er gruppens vurdering at denne sprintlængde har været passende.
Der har i gruppen været givet udtryk for at de store skift, i hvor mange dage en arbejdsuge strækker sig over, kan være forvirrende.
Det har dog ikke været muligt at ændre, da et fast timetal har været nødvendigt for bedre at kunne vurdere hvor meget der kan produceres i løbet af et sprint.
Samtidig giver et svingende antal forelæsninger mm. en ustabilitet i længden af sprint.

\paragraph{Burndown chart}
Gruppen har løbende opdateret et burndown chart der beskriver hvor meget der er produceret i forhold til det forventede.
Flere opgaver er først blevet afsluttet 2-3 dage inde i et sprint.
Som følge af de korte sprint, har dette betydet at gruppen ofte kun har kunnet anvende informationen fra burndown chart'et i sprintets sidste 2-3 dage.

\subsection{Planlægning}
Som et vigtigt led i planlægningen af et sprint, introducerede gruppen planning poker til estimering af opgaver sværhedsgrad/varighed.
I gruppen har planning poker haft den sideeffekt at store uenigheder vedr. længden af en opgave typisk har ført til en længere diskussion af opgavens præcise indhold.
Herved har gruppens medlemmer bedre kunnet sikre en fælles forståelse af den enkelte opgave.
Samtidig har planning poker hjulpet gruppen med at undgå store opgaver ved at kunne indikere hvilke opgaver der skal opdeles i flere små opgaver.

\subsection{Møder}
Som det fremgår af \cref{workshop1:ceremonies} har gruppen afholdt en række møder som del i struktureringen af arbejdsgangen.
Herunder gives en opsummering på udbyttet af disse møder.

\subsubsection{Morgenmøder}
De daglige møder i gruppen har givet positivt afkast på to forskellige måder:
\begin{itemize}
\item De enkelte gruppemedlemmer har opnået et bedre overblik over projektets status, samt hvilke opgaver den resterende gruppe har arbejdet på.
Hermed er eventuelle problemer med opgaver blevet belyst, således at hele gruppen er opmærksomme på eventuelle problemer der skal løses.

\item Gruppens medlemmer har i kraft af en fast dagsorden hurtigere påbegyndt den enkelte dags arbejde efter endt morgenmøde.
Uden denne enkle strukturerede start på dagen, har det været op til det enkelte gruppemedlem at påbegynde sit arbejde.

\end{itemize}

\subsubsection{Fredagsmøder}\label{workshop3:fredag}
Den uformelle (men strukturerede) mødeform til de faste fredagsmøder har styrket gruppen socialt.
Dette især ved at give plads til belysning af arbejds-problemer der ikke er af faglig karakter.
Herved har de enkelte gruppemedlemmer kunnet give udtryk for problemer med besteme værktøjer eller metoder, således at disse problemstillinger også er kendt af den resterende gruppe.

Møderne har desuden ladt gruppens medlemmer give hinanden positiv og negativ kritik.
På denne måde vurderer gruppen at have undgået sociale problemstillinger ved at håndtere dem i fællesskab inden de har kunnet vokse i størrelse.

\subsubsection{Sprint planning}
Som et nødvendigt led i anvendelsen af Scrum er hvert sprint indledt med et planlægnings møde.
Eftersom disse møder har haft til formål at lade gruppen prioritere opgaver før estimering, har de ledt til en kontinuerlig refleksion over hvad det præcise mål for projekt har været.
Netop dette har gjort det muligt for gruppen at sikre, at kernefunktionalitet har haft højest prioritet til trods for skift i problemstilling.

\subsubsection{Sprint review}
Ved efter hvert sprint at gennemgå de udførte opgaver er det lykkedes gruppen bedre at foretage vidensdeling.
Eftersom reviewet udføres af alle gruppemedlemmer, efter alle sprints og med udgangspunkt i de opgaver der er udført, sikres det at alle får kendskab til alle dele af systemet/rapporten.
Denne form for vidensdeling kunne ikke være opnået på morgenmødet alene.

\subsection{Kvalitetssikring}
Som et led i udførelsen af en opgave, når opgaven en review fase.
Her er opgaven blevet gennemgået af et andet gruppemedlem, der har foretaget review og den eventuelle omskrivning af dele af opgaven.
Dette er den eneste form for kvalitetssikring gruppen har foretaget på det producerede.
Som kvalitetssikring af rapport har dette været en god løsning, men i relation til kode kunne andre værktøjer være anvendt.

\subsubsection{Test af kode}
\subsubsection{Pair programming}