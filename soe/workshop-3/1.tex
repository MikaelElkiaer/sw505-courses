\section{Opsummering}\label{workshop3:opsummering}
%What works effectively in your development approach and what is still a problem?
I gruppens daglige arbejde har introduktionen af XP/Scrum været den væsentligste ændring.
Den anvendte metode (som beskrevet i \cref{ourmethod}) har i høj grad medvirket til, at gruppen har oplevet en mere struktureret arbejdsgang.
Som biprodukt af dette har gruppens medlemmer fået større overblik over kommende, samt igangværende, opgaver.
I de følgende afsnit opsummeres på gruppens valgte arbejdsmetode.
Herefter (\cref{konsekvenser} og \cref{workshop3:fremtidig_brug}) diskuteres konsekvenserne af de indførte metoder.

\paragraph{Fredagsmøder}
Ved gruppens fredagsmøder (se \cref{friday_meeting}) har gruppen haft mulighed for løbende at evaluere de metoder der har været taget i brug.
På denne måde er metoder blevet evalueret løbende af alle gruppens medlemmer, såfremt der har været kommentarer til metoderne.
\Cref{workshop3:fredag} giver en yderligere beskrivelse af fredagsmøderne.

\subsection{Brugen af sprints}
Som en del af en agil arbejdsgang var gruppen nødt til at vurdere en passende længde af et sprint.
Længden af sprints blev her vurderet til længden af en fuld arbejdsuge, altså 36 arbejdstimer.
Det er gruppens vurdering at denne sprintlængde har været passende.
Der har i gruppen været givet udtryk for, at de store skift i hvor mange dage en arbejdsuge strækker sig over, kan være forvirrende.
Det har dog ikke været muligt at ændre, da et fast timetal har været nødvendigt for bedre at kunne vurdere hvor meget der kan produceres i løbet af et sprint.
Samtidig giver et svingende antal forelæsninger mm. en ustabilitet i længden af sprint.
%Works effectively:
%	a. Daily Scrum
%	b. Sprints
%		i. Sprint længden var… GOD!!!! :O
%	c. Små, veldefinerede opgaver
%	d. Faste møder
\label{workshop3:fredag}
%		i. Sprint review
%		ii. Sprint planning
%		iii. Fredagsmøde
%		iv. Planning poker
%	e. Fruity programming (sort of)
%
%Still a problem:
%	1. Usikkerhed ift. kvalitetssikring af kode
%
%Indifferents:
%	1. Burndown chart