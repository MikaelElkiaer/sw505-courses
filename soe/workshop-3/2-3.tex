\section{Konsekvenser}\label{konsekvenser}
At arbejde målrettet med at udvikle vores process har givet mange gode erfaringer og ikke mindst optimering af hvordan vi som gruppe kan arbejde mere effektivt sammen.
Idéerne fra de første to workshops er blevet implementeret, og de fleste er også efterfølgende blevet en naturlig del af arbejdsdagen.

\subsection{Konsekvens af valg af værktøjer, teknikker og praksisser}\label{konsekvens:valg}
Som beskrevet i \cref{workshop3:opsummering}, er der blevet inddraget mange nye elementer i gruppens udviklingsprocess, hvor de fleste virker effektivt mens andre ikke har virket i praksis.

Som en konsekvens af elementer i SWOT-analysen omhandlende kvalitet af kode, afholdte vi en workshop omkring Design Patterns (\cref{workshop2:designpatterns}), som skulle afhjælpe de problemer de kunne være med det skrevne kode.

På trods af at de undersøgte mønstre ikke er blevet brugt aktivt under udviklingen, har de alligevel formået at give gruppen et ''fælles sprog'' til at beskrive nogle almindeligt brugte mønstre brugt i software.
Denne viden kan bruges i de situationer hvor der diskuteres mulige løsninger af et givent problem.
En anden konsekvens af introduktionen af Design Patterns har også delvist gjort det muligt at strukturere sin tilgang til problemløsning, da man har fået kendskab til mange afprøvede teknikker.

\subsection{Løsning af problemer}\label{konsekvens:problemer}
Det er endnu uvist om introduktionen af Design Patterns faktisk har afhjulpet de problemer der blev fundet ud fra SWOT-analysen (\cref{swot:weaknesses}).
Dette kan skyldes at det blev gjort forholdsvist sent i projektet, og at vi derfor kun har haft mulighed for at benytte dem i ganske kort tid.
Den umiddelbare indvirkning (\cref{konsekvens:valg}) er ikke særlig målbar, og kan derfor overvejende ses som gruppens antagelser samt det faktum at den i forvejen udviklede kode ikke er blevet refaktoreret mht. Design Patterns.

\section{Fremtidig brug af metode(r)}\label{workshop3:fremtidig_brug}
Der er stor enighed i gruppen om at disse workshops har tilføjet mange positive elementer til projektarbejdet.
I tidligere projekter har vi arbejdet struktureret, uden egentlig at være det, hvor den nye metode har skabt mere ro og overblik i det daglige arbejde.

\subsection{Videre herfra}
Udviklingen af den beskrevne metode i de 3 workshops har givet en ny erfaring samt erkendelse af at den benyttede metode nødvendigvis ikke er den som fungerer bedst.
Fremtidig projektarbejde vil derfor også indeholde mange elementer herfra; bl.a. faste møder, SCRUM board og små veldefinerede opgaver, da disse elementer (blandt andre) har givet et rigtigt godt overblik.

Dog har vi også erkendt at ikke alle elementer i metoden (og mere generelt, også andre metoder) ikke altid passer lige godt på alle dele af et projekt.
Et godt eksempel på dette er at vores metoder fokuserer meget på udvikling (programmering), hvilket ikke altid kan overføres til selve rapportskrivningen, som har en tendens til at følge vandfaldsmetoden, hvilket kan give en konflikt ift. at arbejde agilt.

\subsection{Behov for forbedring}
På trods af den generelle tilfredshed med metoden findes der stadig delelementer som kan forbedres (eller erstattes med andre, og måske mere passende metoder).

Et meget godt eksempel er det faktum at vi fra begyndelsen havde et ønske om at arbejde testdrevet, hvilket ikke har været muligt at følge.
Den primære grund har været det \textit{overhead} det ville tilføje til udviklingen, men også at ganske få i gruppen har erfaring med at arbejde på denne måde.
Konsekvensen af dette har også været at der ikke forelægger nogle klart definerede retningslinjer for test af produktet, så Unit Testing heller ikke er en fast del af udviklingen.

Resultatet af dette er at der stadig ikke eksisterer en metode/praksis til kvalitetssikring af produktet.
