\section{Konsekvenser}\label{konsekvenser}
At arbejde målrettet med at udvikle vores process har givet mange gode erfaringer og ikke mindst optimering af hvordan vi som gruppe kan arbejde mere effektivt sammen.
Idéerne fra de første to workshops er blevet implementeret, og de fleste er også efterfølgende blevet en naturlig del af arbejdsdagen.

\subsection{Konsekvens af valg af værktøjer, teknikker og praksisser}\label{konsekvens:valg}
Som beskrevet i \cref{workshop3:opsummering}, er der blevet inddraget mange nye elementer i gruppens udviklingsprocess, hvor de fleste virker effektivt mens andre ikke har virket i praksis.

Som en konsekvens af elementer i SWOT-analysen omhandlende kvalitet af kode, afholdte vi en workshop omkring Design Patterns (\cref{workshop2:designpatterns}), som skulle afhjælpe de problemer de kunne være med det skrevne kode.

På trods af at de undersøgte mønstre ikke er blevet brugt aktivt under udviklingen, har de alligevel formået at give gruppen et ''fælles sprog'' til at beskrive nogle almindeligt brugte mønstre brugt i software.
Denne viden kan bruges i de situationer hvor der diskuteres mulige løsninger af et givent problem.
En anden konsekvens af introduktionen af Design Patterns har også delvist gjort det muligt at strukturere sin tilgang til problemløsning, da man har fået kendskab til mange afprøvede teknikker.

\subsection{Løsning af problemer}\label{konsekvens:problemer}
Det er endnu uvist om introduktionen af Design Patterns faktisk har afhjulpet de problemer der blev fundet ud fra SWOT-analysen (\cref{swot:weaknesses}).
Dette kan skyldes at det blev gjort forholdsvist sent i projektet, og at vi derfor kun har haft mulighed for at benytte dem i ganske kort tid.
Den umiddelbare indvirkning (\cref{konsekvens:valg}) er ikke særlig målbar, og kan derfor overvejende ses som gruppens antagelser samt det faktum at den i forvejen udviklede kode ikke er blevet refaktoreret mht. Design Patterns.