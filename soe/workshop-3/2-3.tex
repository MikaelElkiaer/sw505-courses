\section{Konsekvenser}\label{konsekvenser}
At arbejde m�lrettet med at udvikle vores process har givet mange gode erfaringer og ikke mindst optimering af hvordan vi som gruppe kan arbejde mere effektivt sammen.
Id�erne fra det f�rste to workshops er blevet implementeret, og de fleste er ogs� efterf�lgende blevet en naturlig del af arbejdsdagen.

\subsection{Konsekvens af valg af v�rkt�jer, teknikker og praksisser}\label{konsekvens:valg}
Som beskrevet i \cref{workshop3:opsummering}, er der blevet inddraget mange nye elementer i gruppens udviklingsprocess, hvor de fleste virker effektivt mens andre ikke har virket i praksis.

Som en konsekvens af elementer i SWOT-analysen omhandlende kvalitet af kode, afholdte vi en workshop omkring Design Patterns (\cref{workshop2:designpatterns}), som skulle afhj�lpe de problemer de kunne v�re med det skrevne kode.

P� trods af at de unders�gte m�nstre er ikke blevet brugt aktivt under udviklingen, har de alligevel form�et at give gruppen et ''f�lles sprog'' til at beskrive nogle almindeligt brugte m�nstre brugt i software.
Denne viden kan bruges i de situationer hvor der diskuteres mulige l�sninger af et givent problem.
En anden konsekvens af introduktionen af Design Patterns har ogs� delvist gjort det muligt at strukturere sin tilgang til probleml�sning, da man har f�et kendskab til mange afpr�vede teknikker.

\subsection{L�sning af problemer}\label{konsekvens:problemer}
Det er endnu uvist om introduktionen af Design Patterns faktisk har afhjulpet de problemer der blev fundet ud fra SWOT-analysen (\cref{swot:weaknesses}).
Dette kan skyldes at det blev gjort forholdsvist sent i projektet, og at vi derfor kun har haft mulighed for at benytte dem i ganske kort tid.
Den umiddelbare indvirkning (\cref{konsekvens:valg}) er ikke s�rlig m�lbar, og kan derfor overvejende ses som gruppens antagelser samt det faktum at den i forvejen udviklede kode ikke er blevet refaktoreret mht. Design Patterns.�